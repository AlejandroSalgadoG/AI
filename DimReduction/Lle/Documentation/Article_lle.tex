\documentclass[12pt,journal]{IEEEtran}

\usepackage[utf8]{inputenc}
\usepackage{graphicx}
\usepackage{amsmath}
\usepackage{amssymb}

\begin{document}

\title{A weighted data representation based approach for manifold learning}
\author{A. Salgado - O. Lucía Quintero M.}
\maketitle

\begin{abstract}
    We implemented the weighted data representation based in KNN and minimization
    of the euclidean distance between representation. Finally the local embedding
    is tested on 3-dimensional manifolds.
\end{abstract}

\section{Introduction}

\section{Theorical framework}

\subsection{First optimization problem}

    \begin{equation*}
        \begin{aligned}
            \underset{w_{ij}}{\text{min}} \quad \sum_{i=1}^n & \lVert x_i - \sum_{j=1}^k w_{ij} N(x_i)_j \rVert^2 \\
            & \sum_{j=1}^k w_{ij} = 1
        \end{aligned}
    \end{equation*}

    \subsection{Case for an arbitrary $i$}

    define

    \[
        V_i = 
        \begin{bmatrix}
            |      &  |     &       & | \\
            N(x_i)_1 & N(x_i)_2 & \dots & N(x_i)_k \\
            |      &  |     &       & |
        \end{bmatrix}_{d \times k}
    \]
    \[
        W_i =
        \begin{bmatrix}
            w_{i1}\\
            w_{i2}\\
            \dots \\
            w_{ik}\\
        \end{bmatrix}_{k \times 1}
        \hspace{1cm}
        e = 
        \begin{bmatrix} 
        1\\
        1\\
        \dots\\
        1
        \end{bmatrix}_{k \times 1}
    \]

    then
    
    \begin{equation*}
        \sum_{j=1}^k w_{ij} N(x_i)_j = V_i W_i
    \end{equation*}

    \begin{equation*}
            \underset{w_{ij}}{\text{min}} \lVert x_i - \sum_{j=1}^k w_{ij} N(x_i)_j \rVert^2
            = 
            \underset{W_i}{\text{min}} \lVert x_i - V_i W_i \rVert^2
    \end{equation*}

    define

    \[
        x_i e^t =
        \begin{bmatrix}
            x_1 \\
            x_2 \\
            \dots \\
            x_d \\
        \end{bmatrix}_{d \times 1}
        \begin{bmatrix}
            1 & \dots & 1
        \end{bmatrix}_{1 \times k}
        =
        \begin{bmatrix}
            |   &       & | \\
            x_i & \dots & x_i\\
            |   &       & |
        \end{bmatrix}_{d \times k}
    \]\\
    
    Now $x_i$ can be defined as

    \begin{equation*}
        x_i = x_i e^t W_i
    \end{equation*}

    Hence

    \begin{equation*}
        \begin{aligned}
            \underset{W_i}{\text{min}} \lVert x_i - V_i W_i \rVert^2
            &=
            \underset{W_i}{\text{min}} \quad \lVert x_i e^t W_i - V_i W_i \rVert^2\\
            &=
            \underset{W_i}{\text{min}} \quad \lVert (x_i e^t - V_i) W_i \rVert^2\\
            &=
            \underset{W_i}{\text{min}} \quad [(x_i e^t - V_i) W_i]^t [(x_i e^t - V_i) W_i]\\
            &=
            \underset{W_i}{\text{min}} \quad W_i^t(x_i e^t - V_i)^t (x_i e^t - V_i) W_i\\
        \end{aligned}
    \end{equation*}

    Define a matrix $G$ as

    \begin{equation*}
        G = (x_i e^t - V_i)^t (x_i e^t - V_i)
    \end{equation*}

    Redefine the problem as

    \begin{equation*}
            \underset{W_i}{\text{min}} \quad W_i^t(x_i e^t - V_i)^t (x_i e^t - V_i) W_i
            =
            \underset{W_i}{\text{min}} \quad W_i^t G W_i
    \end{equation*}

    To solve this problem the Lagrange multipliers are used. First redefine the
    restriction as

    \begin{equation*}
        \sum_{j=1}^k w_{ij} = 1 = e^t W_i
    \end{equation*}

    Now

    \begin{equation*}
        \begin{aligned}
            \mathcal{L}(W_i, \lambda) &= W_i^t G W_i - \lambda (e^t W_i - 1)\\
            \frac{\delta \mathcal{L}}{\delta W_i} &= 2 G W_i - \lambda e
        \end{aligned}
    \end{equation*}

    \begin{equation*}
        \begin{aligned}
            2 G W_i - \lambda e &= 0 \\
            2 G W_i &= \lambda e \\
            G W_i &= \frac{1}{2} \lambda e \\
            G^{-1} G W_i &= G^{-1} \frac{1}{2} \lambda e \\
            W_i &= \frac{1}{2} \lambda G^{-1} e \\
        \end{aligned}
    \end{equation*}

    Now if $\lambda$ is wrongly chosen, then the result would be a scaled version
    of the true value. Then, $\lambda$ can be chosen arbitrarily, and after $W_i$
    is computed a rescale procedure can be conducted in order to satisfy the
    restriction.\\

    For simplicity choose $\lambda = 2$, then

    \begin{equation*}
        \begin{aligned}
            W_i &= \frac{1}{2} \lambda G^{-1} e = G^{-1} e \\
            G W_i &= e \\
        \end{aligned}
    \end{equation*}

    So the problem is reduced to solve a linear system of equations. Now if
    after build the matrix G, it turns out that is a singular matrix the
    following computation can be used to make it invertible

    \begin{equation*}
            G = G + \sigma I
    \end{equation*}

    where sigma is a small value.

    \subsection{Second optimization problem}

    \begin{equation*}
        \begin{aligned}
            \underset{y}{\text{min}} \quad \sum_{i=1}^n & \lVert y_i - \sum_{j=1}^n w_{ij} y_j \rVert^2 \\
            & YY^t = I
        \end{aligned}
    \end{equation*}

    Where

    \[
        Y = 
        \begin{bmatrix}
            |   &  |  &       & |   \\
            y_1 & y_2 & \dots & y_n \\
            |   &  |  &       & |
        \end{bmatrix}_{p \times n}
        \hspace{0.5cm}
    \]

    Now, define

    \[
        W = 
        \begin{bmatrix}
            |   &  |  &       & |   \\
            w_1 & w_2 & \dots & w_n \\
            |   &  |  &       & |
        \end{bmatrix}_{n \times n}
        where
        \hspace{0.5cm}
        W_i = 
        \begin{bmatrix}
            0 \\
            \vdots \\ 
            w_1 \\
            \vdots \\ 
            w_k \\
            \vdots \\ 
            0 \\
        \end{bmatrix}_{n \times 1}
    \]

    Where $W_i$ is a vector that contains the value of the weight that correspond
    to the $j$ neighbor of $x_i$ in the corresponding position, and 0 anywhere
    else. This means that 

    \begin{equation*}
        X W_i = \sum_{j=1}^k w_{ij} x_j
    \end{equation*}

    Rewrite the problem as

    \begin{equation*}
            \underset{y}{\text{min}} \quad \sum_{i=1}^n \lVert y_i - \sum_{j=1}^n w_{ij} y_j \rVert^2
            =
            \underset{y}{\text{min}} \quad \sum_{i=1}^n \lVert y_i - Y W_i \rVert^2
    \end{equation*}

    Also define 

    \[
        I_i = 
        \begin{bmatrix}
            0 \\
            \vdots \\ 
            1 \\
            \vdots \\ 
            0 \\
        \end{bmatrix}_{n \times n}
        \hspace{0.5cm}
    \]

    Which is a vector that has a $1$ in the $i$ position and $0$ anywhere else.
    With this new vector we can redefine the problem as

    \begin{equation*}
            \underset{y}{\text{min}} \quad \sum_{i=1}^n \lVert y_i - Y W_i \rVert^2
            =
            \underset{y}{\text{min}} \quad \sum_{i=1}^n \lVert YI_i - Y W_i \rVert^2
    \end{equation*}

    And now, since this expression is adding the norm of each vector in the
    matrix, the optimization problem can be rewritten to minimize the norm of
    the entire matrix

    \begin{equation*}
            \underset{y}{\text{min}} \quad \sum_{i=1}^n \lVert YI_i - Y W_i \rVert^2
            =
            \underset{Y}{\text{min}} \quad \lVert YI - Y W \rVert^2
    \end{equation*}

    So, this optimization problem can be solved as follows

    \begin{equation*}
        \begin{aligned}
            \lVert Y I - Y W \rVert^2 &= \lVert Y (I - W) \rVert^2\\
            &= Tr[(Y [I-W])^t (Y [I-W])]\\
            &= Tr[(I-W)^t Y^t Y (I-W)]\\
            &= Tr[(I-W)^t Y^t Y (I-W)]\\
            &= Tr[(I-W) (I-W)^t Y^t Y]\\
            &= Tr[Y (I-W) (I-W)^t Y^t]\\
        \end{aligned}
    \end{equation*}

    At this point define a matrix $M$ as

    \begin{equation*}
        M = (I-W) (I-W)^t
    \end{equation*}

    Redefine the optimization problem as

    \begin{equation*}
        \begin{aligned}
            \underset{Y}{\text{min}} \quad \lVert YI - Y W \rVert^2
            &= 
            \underset{Y}{\text{min}} \quad Tr(Y M Y^t)\\
            &=
            \underset{Y}{\text{min}} \quad Tr(M Y^t Y)
        \end{aligned}
    \end{equation*}

    To solve this problem the Lagrange multipliers are used.

    \begin{equation*}
        \begin{aligned}
            \mathcal{L}(Y, \lambda) &= Tr( M Y^t Y) - \lambda (Y Y^t - I)\\
            \frac{\delta \mathcal{L}}{\delta W_i} &= Y M^t + Y M - 2 \lambda Y
        \end{aligned}
    \end{equation*}

    We know that matrix $M$ is symmetric 

    \begin{equation*}
        \begin{aligned}
            M^t &= [(I-W) (I-W)^t]^t\\
                &= [(I-W)^t]^t (I-W)^t\\
                &= (I-W) (I-W)^t\\
                &= M
        \end{aligned}
    \end{equation*}

    Hence

    \begin{equation*}
        \begin{aligned}
            \frac{\delta \mathcal{L}}{\delta W_i} &= Y M^t + Y M - 2 \lambda Y\\
                                                  &= Y M + Y M - 2 \lambda Y\\
                                                  &= 2 Y M - 2 \lambda Y \\
                              2 Y M - 2 \lambda Y &= 0 \\
                                            2 Y M &= 2 \lambda Y \\
                                              Y M &= \lambda Y \\
                                              (Y M)^t &= (\lambda Y)^t \\
                                              M^t Y^t &= Y^t \lambda \\
                                              M Y^t &= \lambda Y^t \\
        \end{aligned}
    \end{equation*}


\section{Algorithm implementation}

\section{Results}

\section{Conclusions}

\end{document}
